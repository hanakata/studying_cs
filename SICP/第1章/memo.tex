\documentclass{jsarticle}
\usepackage{amsmath}
\begin{document}


\section{はじめに}
学ぶもの=計算プロセス

$計算プロセス \xrightarrow{進化} データの操作$

プログラムを作る=プロセスを指揮する

\section{Lisp プログラミング}

プロセスを記述するのには、適切な言語が必要

今後、Lispを使用して説明

\begin{enumerate}
  \item プログラミング要素

強力なプログラミング言語はコンピュータにタスクの実行を指示するだけでなくプロセスについて考えをまとめるのにも使える。

そのため言語を記述する場合には簡単な考えを組み合わせるより複雑な考えを作るためにその言語がどのような手段を提供しているのか注意を払う必要がある。

言語は複雑な考えをつくるために以下のメカニズムを持っている。
\begin{itemize}
  \item 基本式

言語に関わる最も単純な実体を表す
  \item 組み合わせ方法

複合要素をより単純なものから構築する方法
  \item 抽象化方法

複合要素に名前をつけ単体として扱うための方法

プログラムで扱う要素=手続き、データ

\end{itemize}
\end{enumerate}

※Lispの書き方が中心だったので割愛

\section{手続きとそれが生成するプロセス}

考慮中のアクションについて結果を思い描く能力は達人プログラマになるために重要なこと


\end{document}